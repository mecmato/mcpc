Magneto torquers

Hacer pasar una corriente por una espira genera un campo magn�tico. Los campos magn�ticos interact�an para generar torques. �ste es el principio de todos los motores modernos. ION utiliza �stos principios en su sistema de control de actitud mediante el uso de magnetotorquers y torque coils (Bobinas de torque). Las bobinas de torque consisten en un cable arrollado dentro del sat�lite para formar un gran bobinado. Esta espira se usa para generar un campo magn�tico, el cual interact�a con el campo geomagn�tico para cambiar la orientaci�n del sat�lite.

Un corriente que fluye por las bobinas de torque generan un momento dipolar magn�tico seg�n la siguiente ecuaci�n: m=NIAa_n

El par�metro A es la secci�n de la bobina, N es el n�mero de vueltas de la bobina, y I es la corriente que pasa por la bonina. La direcci�n del momento diporlar es normal a la bobina, y queda determinado por la regla de la mano derecha de acuerdo a la direcci�n en la que la corriente atraviesa el bobinado (fig. 2). El campo magn�tico en el centro geom�trico de �ste rect�ngulo, bm, de largo a y ancho b es dado por la siguiente ecuaci�n:
 
